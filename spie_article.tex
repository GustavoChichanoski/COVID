%% spie.cls - Version 3.40
%% LaTeX2e class file to format manuscript for SPIE Proceedings
%% See end of file for history of changes
%
%  The following commands will not be understood in other classes:
%  \authorinfo{}, \keywords{}
%
\NeedsTeXFormat{LaTeX2e}
\ProvidesClass{spie}[2015/08/14  v3.4 SPIE Proceedings class]
\LoadClassWithOptions{article}  % build on standard article class

\DeclareOption{a4paper}{%
  \AtEndOfClass{%
    % \oddsidemargin -0.61cm    % for side margins of 1.93 cm
    % \evensidemargin -0.61cm   % for side margins of 1.93 cm
    \typeout{a4paper used}
  }
}

\DeclareOption{10pt}{\PassOptionsToClass{10pt}{article}}
\DeclareOption{11pt}{\PassOptionsToClass{11pt}{article}}
\DeclareOption{12pt}{\PassOptionsToClass{12pt}{article}}

\ProcessOptions\relax      % set margins for a4paper if specified

%% type out specified font size
\if0\@ptsize\typeout{ten-point font} \fi
\if1\@ptsize\typeout{eleven-point font} \fi
\if2\@ptsize\typeout{twelve-point font} \fi

\RequirePackage{authblk}
\RequirePackage{ifthen}
\RequirePackage{sectsty}
\RequirePackage[superscript]{cite}[2003/11/04] % need vers. > 4.01

\sectionfont{\bfseries\Large\raggedleft}

%% page format (see "Sample manuscript showing specifications and style")
%% following based on default top and left offset of 1 inch = 25.4mm
\topmargin 0.0in                % for top margin of 1.00in
%% the next two side margins are for US letter paper
%% and are overridden by the a4paper option
\oddsidemargin -0.125in           % for side margin of 0.875 in
\evensidemargin -0.125in          % for side margin of 0.875 in
%
\textheight 8.74in                % approx 22.2 cm
\textwidth 6.75in                 % approx 17.1 cm
\headheight 0in \headsep 0in      % avoid extra space for header
\pagestyle{empty}                 % no page numbers is default
\setlength{\parskip}{1ex plus 1ex minus 0.3ex} % spacing between paragraphs
\date{}                           % avoid date

%%  space for floats - figures and tables
\setlength{\abovecaptionskip}{0ex}
\setlength{\floatsep}{0.9ex plus 0.3ex minus 0.6ex}
\setlength{\textfloatsep}{4ex plus 3ex minus 1.5ex}
\renewcommand{\textfraction}{0.10}
\renewcommand{\floatpagefraction}{0.60}
\renewcommand{\topfraction}{0.90}
\renewcommand{\bottomfraction}{0.90}
\setcounter{totalnumber}{3}
\setcounter{topnumber}{2}
\setcounter{bottomnumber}{2}

%%%%%%%%%%%%%%%%%%%%
%%%%  define \ample font size %%%%
%% 10% larger than \normalsize for 10 pt,
%% but smaller than \large
\def\ample{\large}   % insurance

\def\setamplefont{
  \if0\@ptsize    % 10pt text; use 11pt bold fonts
    \def\ample{\@setsize\large{12pt}\xipt\@xipt}
  \fi

  \if1\@ptsize    % 10pt text; use 11pt bold fonts
    \def\ample{\@setsize\large{12pt}\xiipt\@xiipt}
  \fi

  \if2\@ptsize    % 12pt text; use 13.2pt bold fonts
    \renewcommand{\ample}{\fontsize{13.2pt}{15.8pt}\selectfont\bfseries}
    \ifcmr \font\ample = cmbx17 at 13.2truept \fi
  \fi
}


%%%% define title aspects %%%%
\def\titlefont{\normalfont\LARGE\bfseries}  %% insurance
\def\title#1{\gdef\@title{\titlefont #1}}

\def\settitlefont{
  %%% identify font used; cmr = Computer Modern Roman, ptm = Times Roman
  \typeout{\rmdefault -font_in_spiecls}
  \def\cmr{cmr}  % variables for font tests
  \newif\ifcmr
  \def\ptm{ptm}
  \newif\ifptm
  \ifthenelse{\equal{\rmdefault}{\cmr}}{\cmrtrue}{\cmrfalse}
  \ifthenelse{\equal{\rmdefault}{\ptm}}{\ptmtrue}{\ptmfalse}
  %%% scale title according to default font and size
  \if0\@ptsize    % 10pt text; use 16pt bold fonts
    \renewcommand{\titlefont}{\fontsize{16pt}{19.2pt}\selectfont\bfseries}
    \ifcmr  \font\titlefont = cmbx12 at 16truept  \fi
  \fi
  % for 11pt text, title font should be 17pt = \LARGE for 11pt. No change needed
  \if2\@ptsize    % 12pt text; use 19.2pt bold fonts
    \renewcommand{\titlefont}{\fontsize{19.2pt}{23pt}\selectfont\bfseries}
    \ifcmr  \font\titlefont = cmbx12 at 19.2truept  \fi
  \fi
}

\def\authorinfo#1{\gdef\@authorinfo{#1}}
\authorinfo{}     %% default is empty
\let\@affiliation\@empty
\def\affiliation#1{\gdef\@affiliation{#1}}

% Set special footnote indentation for authorinfo
\RequirePackage{authblk}
\renewcommand*{\Authsep}{, }
\renewcommand*{\Authand}{, }
\renewcommand*{\Authands}{, }
\renewcommand\@makefntext[1]{\leftskip=1.4em\hskip-0em\@makefnmark#1}


\def\maketitle{\par
  \settitlefont
  \setamplefont
  \begingroup
    \def\thefootnote{\fnsymbol{footnote}}%
    \def\@makefnmark{\hbox
        to\z@{$\m@th^{\@thefnmark}$\hss}}%
    \if@twocolumn
      \twocolumn[\@maketitle]%
    \else \newpage
      \global\@topnum\z@
      \@maketitle
    \fi
    \@thanks
  \endgroup
  \let\maketitle\relax
  \let\@maketitle\relax
  \gdef\@thanks{}\gdef\@author{}\gdef\@title{}\let\thanks\relax

  %%%%  define footnote attributes %%%%
  \renewcommand{\footnotesize}{\small}  % enlarge footnote font to small
  \renewcommand{\thefootnote}{\fnsymbol{footnote}}
  \ifx\@authorinfo\empty \else\footnotetext[0]{\@authorinfo}\fi
  %% NB use zero to avoid footnote mark
  %% use footnote symbols, not numbers
  \renewcommand{\thefootnote}{\fnsymbol{footnote}}

  % Set footnote indentation back to standard
  \renewcommand\@makefntext{\leftskip=0em\hskip1.4em\@makefnmark}
}

% redefine \and for author list because \tabular was removed
% \def\and{\bigskip\\}

\def\@maketitle{\newpage
  \null
  \vspace{-5}
  % logo
  \vspace{-20mm}
  \begin{figure}[htb]
      \centering
      \includegraphics[scale=.5]{cbsIII.png}
      \label{fig:my_label}
  \end{figure}
  \vspace{.5cm}
  % move title to top of page
  \if0\@ptsize\vspace{-10mm}\else\vspace{-12mm}\fi
  \begin{center}%
  \ifcmr
      \if0\@ptsize {\typeout{10pt}\setlength{\baselineskip}{19.2pt} \@title \par} \fi
      \if1\@ptsize {\typeout{11pt}\setlength{\baselineskip}{20.4pt} \@title \par} \fi
      \if2\@ptsize {\typeout{12pt}\setlength{\baselineskip}{23pt} \@title \par} \fi
  \else
      %{\@title \par}
  \fi
      % Adicionar nomes
      {\setlength{\baselineskip}{4.3ex} \@title \par}
      %   \vskip 3.5mm
      %   {\large        % author and organization font size
      %   \@author}   % remove tabular used in article.cls
      %   \vskip 1.5ex
      %   {\large \@date}%
  \end{center}%
  \par
}

%% Bibtex

%%%% section aspects %%%%
% all headings bold
% center section headings, sample size
\def\sectfont{\ample\bf}
% sub- and subsubsection headings flush left
\def\subsectfont{\ample\bf}
\def\subsubsectfont{\sectfont\normalsize\bf}
\def\append{0}

\def\section{\@startsection{section}{1}{\z@}
   {-2.5ex plus -1ex minus -0.5ex}{0.2ex plus 0.5ex minus 0ex}{\sectfont}}
\def\subsection{\@startsection{subsection}{2}{\z@}
   {-1.5ex plus -1ex minus -0.5ex}{0.1ex plus 0.1ex minus 0.1ex}{\subsectfont}}
\def\subsubsection{\@startsection{subsubsection}{3}{\z@}
   {-1ex plus -1ex minus -0.5ex}{0.1ex plus 0.1ex}{\subsubsectfont}}

%% from latex.sty
%% \@sect{NAME}{LEVEL}{INDENT}{BEFORESKIP}{AFTERSKIP}
%% {STYLE}[ARG1]{ARG2}
\def\@sect#1#2#3#4#5#6[#7]#8{
  \ifnum #2>\c@secnumdepth
    \let\@svsec\@empty
    \let\@svsecp\@empty
    \let\@svsub\@empty
  \else
    \refstepcounter{#1}
    \edef\@svsec{\csname the#1\endcsname\hskip 0.5em plus 0.3em}
    \edef\@svsecp{\csname the#1\endcsname.\hskip 0.3em plus 0.3em}
    \edef\@svsub{\csname the#1\endcsname\hskip 0.5em plus 0.3em}
  \fi
  \@tempskipa #5\relax
  \ifdim \@tempskipa>\z@
    \begingroup #6\relax
      \ifnum #2=1
        %%(kmh) in appendix, add word appendix in front of section number
        \ifnum \append=1 {\interlinepenalty \@M
          APPENDIX \@svsecp\uppercase{#8}\par}
        \else {\interlinepenalty \@M \@svsecp\uppercase{#8}\par}
        \fi
      \else
        \ifnum #2=2
          \noindent{\interlinepenalty \@M \@svsub #8\par}
        \else
          \noindent{\interlinepenalty \@M \@svsub #8\par}
        \fi
      \fi
    \endgroup
    \csname #1mark\endcsname{#7}
    \addcontentsline{toc}{#1}{
      \ifnum #2>\c@secnumdepth
      \else \protect\numberline{\csname the#1\endcsname}
      \fi
    #7
    }
  \else
    \def\@svsechd{
      #6\hskip
      #3\relax  %% \relax added 2 May 90
      \@svsec
      #8\csname
      #1mark\endcsname
      {#7}\addcontentsline
      {toc}{#1}{
        \ifnum #2>\c@secnumdepth 
        \else
          \protect\numberline{\csname the#1\endcsname}
        \fi
        #7}
    }
  \fi
  \@xsect{#5}
}

%%%%% Special sections %%%%%
\def\abstract{\section*{RESUMO}}
\def\endabstract{}

% Keywords
\def\keywords#1{
\par\vspace{0.5ex}{\noindent\normalsize\bf Palavras-chave:} #1
\vspace{0.5ex}   %% provide extra space before first section
}

\def\acknowledgments{\section*{ACKNOWLEDGMENTS}}
\def\endacknowledgments{}
% include old spelling - which is acceptable, but not preferred
\def\acknowledgements{\section*{ACKNOWLEDGMENTS}}
\def\endacknowledgements{}

%%%% Add theorem, lemma, and definition environments %%%%
% kmh - noindent
\def\@begintheorem#1#2{
   \par\noindent\bgroup{\sc #1\ #2. }\it\ignorespaces}
\def\@opargbegintheorem#1#2#3{
   \par\bgroup{\sc #1\ #2\ (#3). }\it\ignorespaces}
\def\@endtheorem{\egroup}
\def\proof{\par{\it Proof}. \ignorespaces}
\def\endproof{{\ \vbox{\hrule\hbox{%
   \vrule height1.3ex\hskip0.8ex\vrule}\hrule
  }}\par}
\newtheorem{theorem}{Theorem}[section]
\newtheorem{lemma}[theorem]{Lemma}
\newtheorem{definition}[theorem]{Definition}

%%%% Figure and table captions %%%
\long\def\@makecaption#1#2{%     % from article.cls
  \vskip\abovecaptionskip
  \sbox\@tempboxa{{\footnotesize #1.\ }{\footnotesize #2}}%
  \ifdim \wd\@tempboxa >\hsize   % with period
    {\footnotesize #1.\ }{\footnotesize #2 \par}
  \else
    \global \@minipagefalse
    \hb@xt@\hsize{\hfil\box\@tempboxa\hfil}%
  \fi
  \vskip\belowcaptionskip}

%%%% appendix aspects %%%%
% use \appendix to start an appendix
% use \section{} for each appendix section
\def\appendix{\def\append{1}
  \par
  \setcounter{section}{0}
  \setcounter{subsection}{0}
  \setcounter{subsubsection}{0}
  \def\thesection{\Alph{section}}
  \def\thesubsection{\Alph{section}.\arabic{subsection}}
\def\thesubsubsection{
\Alph{section}.\arabic{subsection}.\arabic{subsubsection}}
}

%%%%%%%%
\def\thebibliography{%
    \section*{\refname}
    \@thebibliography}
    \let\endthebibliography=\endlist
    \def\@thebibliography#1{\@bibliosize
    \list{\@biblabel{\arabic{enumiv}}}{\settowidth\labelwidth{\@biblabel{#1}}
    \if@nameyear
    \labelwidth\z@ \labelsep\z@ \leftmargin\parindent
    \itemindent-\parindent
    \else
    \labelsep 3\p@ \itemindent\z@
    \leftmargin\labelwidth \advance\leftmargin\labelsep
    \fi
      \itemsep \z@ \@plus 0.5\p@ \@minus 0.5\p@
    \usecounter{enumiv}\let\p@enumiv\@empty
    \def\theenumiv{\arabic{enumiv}}}%
    \tolerance\@M
    \hyphenpenalty=50
     \hbadness5000 \sfcode`\.=1000\relax
}
\newcommand\newblock{\hskip .11em\@plus.33em\@minus.07em}
\@namedef{thebibliography*}{\section*{}\@thebibliography}
\@namedef{endthebibliography*}{\endlist}
\if@nameyear
  \def\@biblabel#1{}
\else
 \def\@biblabel#1{[#1]\hskip \z@ \@plus 1filll}
\fi
\newcount\@tempcntc
\def\@citex[#1]#2{\if@filesw\immediate\write\@auxout{\string\citation{#2}}\fi
 \@tempcnta\z@\@tempcntb\m@ne\def\@citea{}\@cite{\@for\@citeb:=#2\do
  {\@ifundefined
   {b@\@citeb}{\@citeo\@tempcntb\m@ne\@citea\def\@citea{,}{\bfseries ?}\@warning
   {Citation `\@citeb' on page \thepage \space undefined}}%
  {\setbox\z@\hbox{\global\@tempcntc0\csname b@\@citeb\endcsname\relax}%
   \ifnum\@tempcntc=\z@ \@citeo\@tempcntb\m@ne
    \@citea\def\@citea{,}\hbox{\csname b@\@citeb\endcsname}%
   \else
    \advance\@tempcntb\@ne
    \ifnum\@tempcntb=\@tempcntc
    \else\advance\@tempcntb\m@ne\@citeo
    \@tempcnta\@tempcntc\@tempcntb\@tempcntc\fi\fi}}\@citeo}{#1}}
\def\@citeo{\ifnum\@tempcnta>\@tempcntb\else\@citea\def\@citea{,}%
 \ifnum\@tempcnta=\@tempcntb\the\@tempcnta\else
  {\advance\@tempcnta\@ne\ifnum\@tempcnta=\@tempcntb \else \def\@citea{--}\fi
   \advance\@tempcnta\m@ne\the\@tempcnta\@citea\the\@tempcntb}\fi\fi}

%% end of spie.cls
